\chapter{Conclusion}

Validating the automatic ABCD rules is challenging because public datasets are scarce and often lack sufficient data. For example, PH$^2$ contains 200 images on asymmetry, colour, and some dermoscopic structures but misses border irregularity. Therefore researchers aiming to measure borders use private or privately annotated datasets. Furthermore, many papers measuring asymmetry, colour and dermoscopic structures lack validation using public datasets despite PH$^2$ being available at the date of their publication. On the other hand, public datasets are crucial to comparing, validating, and reproducing algorithms. Therefore ABCD rules (apart from the border) will be validated using PH$^2$ datasets so that future researchers can replicate techniques. Furthermore, once rules are combined using Bayesian fusion, a type of probabilistic analysis, results can conform to the diagnosis between malignant and benign, validated from larger datasets, including ISIC 2019.

Finding the border cut-off is fundamental for the classification of melanoma using the ABCD rules\cite{Pereira2020}. Many valuable techniques use statistical models, including LBPC and Otsu, instead of transposed CNNs such as SegNet. Hybrid approaches using SegNet followed by Otsu to measure the border cut-off have been proven beneficial. However, using SegNet without a statistical model is worse when used with the ABCD rules than current methods such as LBPC and Otsu. Therefore, exploring other statistical segmentation techniques and hybrids would be beneficial. Furthermore, segmentation ground-truths do not always correspond to good classification accuracy with ABCD rules, which means even a low accuracy segmentation compared to datasets might have better accuracy when classifying the ABCD rules for border irregularity.

Statistical models for asymmetry, border, and colour extract relevant features for melanoma classification. The goal is to mimic the diagnostic procedure that clinicians are familiar with to produce results that they can utilise in a clinical environment. Extracting relevant features using box-counting and bi-folds ensures capturing relevant features and that the technique is retractable. However, accuracy is lacking in these techniques where superpixels improved asymmetry, changing the accuracy from 58.5\% to 61\% for the PH$^2$ dataset. Further improvements will be made after training an SVM model using the extracted features. Further implementation of convexity and Zernike moments for border irregularity will improve the accuracy. Furthermore, implementing a texture comparison for asymmetry measurements improve accuracy again.

\chapter{Future Work}
Developing algorithms to extract features of ABCD rules is beneficial to GPs because it improves interpretability. Future work will involve extracting more features and training SVM models. For example, extracting more relevant asymmetry features will help classify asymmetry as there is currently no unification of shape, colour, and texture into a single classification model. The extracted features will be combined into a diagnosis between benign and malignant using a Bayesian probabilistic network. Bayesian probability is beneficial because its highly accurate\cite{Takruri2017} and modifiable and ability to classify with incomplete data. For example, asymmetry, border, and colour are sometimes enough to classify skin lesions. However, in some cases, dermoscopic structures or other meta-data, including age, gender, touch, feeling, and location on the body, are required for an accurate diagnosis. Furthermore, This might benefit GPs because it encourages considering a wide range of not always considered features.

Melanoma evolves from benign lesions at initially 30\%-50\%, and despite its significance, clinicians or computers are not yet able to reliably predict this change. AI trained on relevant images could predict melanoma before it occurs\cite{Sondermann2019}. Data on skin lesion evolution is rare in public datasets. However, the associated organisation has taken images of the same skin lesion multiple times. It would be incredibly beneficial to assess the quality of these images, which could potentially lead to the development of a technique describing evolution. Considering evolution in machine learning techniques in the future would be incredibly beneficial to the early detection of melanoma but can only be achieved when there is more data.