\chapter{Analysis of Explainability for the Detection of Melanoma}

\section{Introduction}
Explainable AI (XAI) techniques have recently gained attention because the European general data protection regulation (GDPR and ISO/IEC 27001) stated that these approaches, commonly referred to as ``black box'' approaches, are challenging to utilise in medical environments. Since then, there has been significant progress in making neural network architectures more interpretable. A wide range of techniques\cite{Fuji2019,  Selvaraju2016, Ribeiro2016} have since been developed, demonstrating that it is possible to make neural network techniques interpretable. However, the problem is instead the current scepticism on whether these techniques are trustworthy\cite{Tjoa2019, Samek2019a}, and they can produce realistic but incorrect results\cite{Ghorbani2019}. Some other interpretable techniques do not utilise neural networks. For example, Javier López-Labraca et al.\cite{Lopez-Labraca2018} described an interpretable technique using multiple SVM models with colour and three dermoscopic structures (i.e., pigment networks, globules, and streaks). Bayesian fusion combines each model to calculate a diagnosis. Bayesian probability is a type of probability theory that uses probability distribution to estimate the values of unobserved variables. Bayesian fusion has a comparable accuracy to neural network techniques\cite{Takruri2017}. Overall, results should be partially interpretable for use within clinical environments.

%Deepshap

%Gradcam

